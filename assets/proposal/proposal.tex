\documentclass{article}

% Packages for equations, fonts, and colors.
\usepackage{mathtools}
\usepackage{amsfonts}
\usepackage{xcolor}

% Packages and Commands for list head.
\usepackage{enumitem}
\renewcommand*{\thesection}{\Roman{section}}
\renewcommand*{\thesubsection}{\thesection.(\alph{subsection})}

% Packages and Commands for size.
\usepackage{geometry}
\geometry{
a4paper,
left=20mm,
right=20mm,
top=20mm,
bottom=20mm
}
% Packages and Commands to set up paragraph indentation.
\usepackage{lipsum}
\usepackage{setspace}
\setlength{\parindent}{0em}
\setlength{\parskip}{1.2em}

% Packages and Commands to set up citation, links, etc.
\usepackage{hyperref}
\hypersetup{
colorlinks=true,
citecolor=blue,
linkcolor=blue,
filecolor=magenta,
urlcolor=blue
}

% Packages and commands for title spacing.
\usepackage{titlesec}
\titlespacing*{\section}
{0cm}{0.1cm}{0.1cm}
\titlespacing*{\subsection}
{0.25cm}{0.1cm}{0.1cm}

% Commands to set up reference template.
\def\BibTeX{{\rm B\kern-.05em{\sc i\kern-.025em b}\kern-.08em T\kern-.1667em\lower.7ex\hbox{E}\kern-.125emX}}

% ------------------------------------------------------------

\title{Project Proposal: Digits Classification using CNN}
\author{Team 5\footnote{Contact: \{shiwechen6-c, sxwang6-c\}@my.cityu.edu.hk}  \:- Shiwei Chen, Shixiang Wang}
\date{March 2022}

\begin{document}

\maketitle

\section{Introduction}
In this project, we will research on classifying handwritten digits, a subtask under the field of Contextual Image Classification in Computer Vision. The task is provided by the famous Modified National Institute of Standard and Technology datasets \cite{ref1} (Hereinafter referred as \textit{the MNIST dataset}). In the early days of machine learning, the dataset was widely used for benmarking various image processing systems. While the original datasets contains $60000$ samples for training and $10000$ samples for testing, our project uses only a fraction of them, which are $2000$ samples for training and $2000$ samples for testing. Additionally, the project will be tested on a hidden dataset at the end of submission deadline to determine the final performance. In the subsequent sections, we will describe our method for the task, the basic structure of the experiment, and the expected outcome.

\section{Methodology}
\section{Experiment Plan}
\section{Expected Result}

\begin{thebibliography}{00}
\bibitem[Y. LeCun, et al, 1998]{ref1} Y. LeCun, L. Bottou, Y. Bengio, and P. Haffner. "Gradient-based learning applied to document recognition." Proceedings of the IEEE, 86(11):2278-2324, November 1998.
\end{thebibliography}


\end{document}